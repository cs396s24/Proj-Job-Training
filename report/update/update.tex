\documentclass[12pt]{article}

\usepackage{amsmath}
\usepackage{amssymb}
\usepackage{enumerate}
\usepackage{enumitem}
\usepackage{booktabs}
\usepackage{csquotes}
\usepackage[margin=2cm]{geometry}
\usepackage{hyperref}
\usepackage{tabularx}
\usepackage{tikz}
\usetikzlibrary{patterns, shapes.geometric, positioning, bayesnet}

\usepackage{titling}
\setlength{\droptitle}{-7em}
\usepackage{titlesec}
\titlespacing\section{0pt}{12pt plus 4pt minus 2pt}{4pt plus 2pt minus 2pt}


\title{Project Update}

\author{CS396 Causal Inference}

\begin{document}

\maketitle

\section*{Instructions}

This assignment is due on Friday, May 17 at 11:59pm CDT. It will accepted up to
24 hours late with a 20\% grade penalty.

Please upload a single PDF for your group to Canvas. You are encouraged to use
the TeX file for this assignment to create and format your PDF. Your project
update should no more than five pages total, not including references. 

As always, your work must be your own. It's fine to use published packages or
preprocessing code, but don't claim credit for work that you didn't do. If you
use information from other sources, you must cite those.

\paragraph{Rubric (10 points total)}

\begin{itemize}
\item (1 point) The update is at most five pages long, not including references
\item (1 point) List your group members
\item (1 point) Describe major changes to your project since your proposal
\item (1 point) Include a causal graph and describe the variables
\item (1 point) Describe and interpret your causal estimand
\item (1 point) Provide and interpret a point estimate of your causal effect
\item (1 point) Produce a synthetic data distribution that matches your assumptions
\item (1 point) Estimate and interpret the causal effect from synthetic data
\item (1 point) Propose at least one next step for the final report
\item (1 point) Ask at least one question you want feedback on
\end{itemize}

\clearpage

\section{Group members}

\noindent Please list your group members.

\section{Big changes}

What's changed in what you're focusing on since you wrote your proposal?  This
is your opportunity to let me know about any revisions to the project you've
made since writing your proposal. If you need help deciding on possible
revisions, please reach out to me before turning this in.

\section{Causal graph} \label{sec:graph}

\noindent Draw a causal graph for your problem.

You can use any reasonable method to produce an image of your graph as long as
it is easy to read.  You could, for example, use the
\href{https://aci-esp.shinyapps.io/shiny-short-example/}{ID algorithm GUI at
this website} or the {\tt tikz} latex package, like below.  Your graph should
use as many variables from your dataset as you can, but you aren't required to
use them all. If you have a single treatment, a single outcome, and many
confounders, you can just include a single confounder node and then indicate
that it represents a vector of variables. For each node in your graph, please
say what column(s) of your dataset it corresponds to, if it isn't obvious.

\vspace{1em}

\begin{tikzpicture}[>=stealth, node distance=2.2cm]
    \tikzstyle{format} = [draw, very thick, circle, minimum size=1.2cm, inner sep=1pt]

    \begin{scope}
        \path[->, very thick]

            node[format] (C) {C}
            node[format, right of=C] (D) {D}
            node[format, below of=C] (A) {A}
            node[format, right of=A] (M) {M}
            node[format, right of=M] (Y) {Y}

            (C) edge[blue] (A)
            (D) edge[blue] (A)
            (C) edge[blue] (Y)
            (D) edge[blue] (Y)
            (A) edge[blue] (M)
            (M) edge[blue] (Y)

        ;
    \end{scope}
\end{tikzpicture}

\section{Counterfactual function} \label{sec:counterfactual}

For your treatment $A$ and outcome $Y$, write out $E[Y^a]$ as a function of the
observed data. For example, in the frontdoor example where we had an unobserved
$U$, we wrote:

\[
E[Y^a] = \sum_m p(M=m \mid A=a) \sum_{a'} E[Y \mid A=a', M=m] p(A=a')
\]

You can find use \href{https://aci-esp.shinyapps.io/shiny-short-example/}{ID
algorithm GUI at this website} to find such a function, but please use LaTeX to
reformat it rather than taking a screenshot.

\subsection{Assumptions}

What are the assumptions required for your function above to be an unbiased
estimator of the causal effect? For example, are you assuming consistency?
Conditional exchangeability? What else? For the conditional exchangeability
statements, write them in the form of $X \perp Y \mid Z$.

\section{Estimation and Interpretation} \label{sec:estimation}

Choose a causal effect to estimate, such as the risk difference or risk ratio.
Describe how you will estimate that (i.e., what models will you fit?).  Will
you use the backdoor or frontdoor estimator, or something else? If so, say so.
If you have a more complicated graph and will need a more complicated
estimator, describe why neither the backdoor nor frontdoor estimator will work.
Rather than trying to develop your own estimator for a more complicated
setting, you can use a package like
\href{https://github.com/y0-causal-inference/y0}{\tt $Y_0$},
\href{https://gitlab.com/causal/ananke}{Ananke}, or
\href{https://microsoft.github.io/dowhy/}{DoWhy}.

\subsection{Point estimate}

Use your estimator to get a single estimate for a causal effect your project
focuses on. If you proposed using several outcomes and treatments in your
proposal, you can just focus on one for now. Explain what approach you took to
estimating the causal effect and share the numerical estimate of that effect.

\subsection{Interpretation}

What does your estimate of the causal effect mean for your problem?  Tie this
back to the reasons why you were interested in this dataset in the first place.
If you were able to make policy decisions based on your analysis of this data,
what would these results tell you?

For example, if your treatment were pet ownership, your outcome were
cardiovascular health, and you chose to estimate the risk ratio, you might say
something like ``our results indicate that if we intervened to require someone
to own a pet, their risk of cardiovascular disease decreases by 10\%.''

\section{Synthetic data} \label{sec:synthetic}

For the real-world dataset you consider, we can't check to see whether your
assumptions are valid or your effect estimate is close to the real answer.  To
convince yourself that the methodology you're using is implemented correctly
and that \emph{if} your assumptions are correct \emph{then} your estimator
should be unbiased, you'll apply your methods to a synthetic dataset. 

\subsection{Synthetic generation}

First, construct a synthetic dataset. This should be as close in format to
your real-world dataset as possible. For example, if your real-world outcome is
continuous, your synthetic outcome should be too. If your real-world dataset
has many confounding variables, be sure to at least include two in your
synthetic data.

Describe your data-generating process. You can do so either in words or as a
code snippet that shows how you sample the data. You can refer to the
\texttt{observed} function in HW1 as a guide.

Generate a dataset with at least 1000 rows that you will use in the next part.

\subsection{Synthetic estimation}

For the dataset you generated, what is the true causal effect of the treatment
on the outcome? In the HW1 example, that was 0.5. Explain how you know what it
is in your setting.

Apply the estimator you used from \S\ref{sec:estimation} to estimate this
causal effect. Do you get (approximately) the right answer? The answer to that
question should be yes; try to debug your methodology or increase your
synthetic data sample size if not.

\section{Next steps for the project} \label{sec:next_steps}

The work above covers the most fundamental aspect of a causal analysis. For
your final project, you might consider trying a different estimator or choosing
a different treatment or outcome to look at other interpretations of your data.

In addition to any such new analyses, you need to consider {\bf at least
one} new methodological considerations that we have or will
cover in the second half of the course. These are listed in Appendix \ref{app:next_steps}.

Pick (at least) one of these challenges and discuss how it would apply to your
project thus far. You don't to have implemented anything by the time you turn
in this assignment, but you should describe what you plan to do.

\section{Ask for feedback}

List at least one part of your project thus far on which you'd specifically like
feedback from me. For example: are you stuck on trying to use an existing
package or aren't sure how to preprocess your data to apply one of the
estimators you wrote? Do you have some initial results but aren't sure how to
move forward on interpreting them?

\section*{References}

Please cite any sources you referenced, including links to your datasets and
any packages you used.

\clearpage

\appendix

\section{Possible directions for next steps} \label{app:next_steps}

Here are a few possible directions you might consider; you need to pick at
least one and discuss in some detail how you hope to do so. {\bf You do not need
to include the text for all these possible directions in your update writeup.}

\paragraph{Unobserved confounding}

Think about possible variables that could be important to your problem but
are not in the dataset you're analyzing. Do such variables introduce confounding
between your treatment and outcome? If so, return to Section
\ref{sec:counterfactual} and propose a new function that would allow you to
handle this confounding, and return to Section \ref{sec:estimation} to see
how it changes your estimates of the causal effect.

\paragraph{Causal discovery}

What if the causal graph you drew in Section \ref{sec:graph} is wrong? Use
a causal discovery algorithm to try to learn the causal graph from your
dataset, and supplement the algorithm with any domain knowledge you're confident in.
\href{https://rawgit.com/cmu-phil/tetrad/development/tetrad-gui/src/main/resources/resources/javahelp/manual/tetrad_tutorial.html}{The
TETRAD library} is commonly used for causal discovery, but can be a bit
difficult to use. I've
\href{https://github.com/zachwooddoughty/causal-learn}{forked the {\tt
causal-learn} library} here (a python port of TETRAD) to which I've added a
``fast conditional independence test (FCIT)'' which can more easily handle a
mix of discrete and continuous data.

If you decide to focus on causal discovery, you might want to explore how well
different methods (e.g. constraint-based versus score-based) work on your
particular data. You can also discuss how does changing your causal graph
change how you'd choose your function in Section \ref{sec:counterfactual} and
your estimation in Section \ref{sec:estimation}.

\paragraph{Measurement error}

What if one or more variables you're using in your causal graph are actually
just noisy proxies for the variables you wish you had? For example, if you're
conditioning on BMI as a covariate, you might consider trying to formulate that
as a mismeasured proxy for obesity. To apply these methods, you might need
to consider whether you can find data to estimate the error rate $p(C^* \mid C)$,
e.g., $p(\text{BMI} \mid \text{Obesity})$.

If there's a machine learning classifier which predicts a variable that would
be helpful for your analysis, you could treat its outputs as a noisy proxy
and use its classification error rate as the $p(C^* \mid C)$.

\paragraph{Double Machine Learning}

What if the relationship between A and Y is confounded by a nonlinear
relationship involving a high-dimensional confounder? Just training a linear
regression $E[Y \mid A, C]$ won't work in general, but so-called double machine
learning provides a way to get unbiased estimates of the causal effect. While
implementing these methods is outside the scope of this class, you can use
existing implementations from the
\href{https://microsoft.github.io/dowhy/}{DoWhy} or similar packages.

\paragraph{Missing data}

We haven't covered this yet, but will do so soon. If you want additional
information on this topic to decide whether it's something you're interested
in, you can see the Files folder on Canvas which has slides from 2022.

If you have missing values in your data, what additional assumptions do you
have to make in order to identify the causal effect? If you only consider rows
that have complete data, you're making an ``MCAR'' assumption that might not be
very believable. Depending on the extent of the missingness, you can approach
this by thinking hard about the underlying graphical model and figuring out an
approach specific to your dataset. On the other hand, you can use
\href{https://github.com/kshedden/mice_workshop}{a general approach like MICE}
to impute the missing values from the observed data. If you decide to focus on
missing data, you'll want to think about what additional assumptions your
chosen method makes, and explore how your estimates in Section
\ref{sec:estimation} change.

\paragraph{Selection bias}

We haven't covered this yet, but will do so soon. If you want additional
information on this topic to decide whether it's something you're interested
in, you can see the Files folder on Canvas which has slides from 2022.

Selection bias affects all datasets in some way, but may affect certain analyses
more than others. If you want to focus on selection bias, think about the
data-generating process -- are there certain kinds of rows that are more likely
to end up in your dataset than ``in the wild?'' If so, what does that selection
`mechanism' depend on?

We talked about two ways to approach selection bias: either try to find
external data on $p(S=1 \mid X)$ to counteract $P(X \mid S=1)$, or target a
causal effect such as the odds ratio that has built-in symmetry between the
treatment and the outcome. Either way, you will have to tweak your function
in Section \ref{sec:counterfactual} and your estimator in Section \ref{sec:estimation}.


\end{document}
